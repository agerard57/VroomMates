\section{Attentes de l'utilisateur}

À l'accès du site par un utilisateur, ce dernier s'attend à pouvoir effectuer une série d'actions. On note particulièrement:

La création d'un compte (via Google ou non) et la possibilité de le modifier par la suite. Mettre une photo de profil, renseigner une biographie descriptive, plusieurs renseignements en ce qui concerne les voyages (bavard ou non, présence de musiques pendant les trajets ou non, etc.), et quelques hobbies.
Pouvoir réserver un trajet (fréquent ou non), en passant par la recherche d'annonces, le paiement, les messages privés pour plus de renseignements, la notation du conducteur (et du passager dans le cas inverse).
La possibilité de visualiser tous les voyages futurs ou passés, que ce soit pour un conducteur ou un passager.
Une page d'administration avec la possibilité de bannir des membres, la présence de diagrammes statistiques importantes sur le site (nombre de comptes par pays, nombre total de voyages), ainsi que la possibilité d'accepter ou de refuser une demande de vérification du permis.

\section{Contraintes du projet}

\subsection{Contrainte technique}
\label{Contrainte technique}

La première contrainte est technique : La structure et les outils qui vont être mis en place peuvent ne pas avoir été utilisés préalablement par nos membres de l'équipe, il faudra donc s'adapter en conséquence grâce à la planification de ce lot 1 et sur les connaissances pré-existantes de l'un et de l'autre.

\subsection{Contrainte de temps}
\label{Contrainte de temps}

Le projet a été lancé le 7 mars 2022. Pour le 19 juillet, nous devions rendre ce premier rapport constituant toutes les étapes de conception et de coordination du projet. Ce temps a l'air amplement suffisant à premier abord. Cependant, quand on se penche un peu plus sur le sujet, on s'aperçoit vite de la quantité de travail que cela demande, à réaliser dans une période de la journée très brève (après le travail ou les cours). Forcément, plus nous avancions dans le temps, plus nous avions de nouvelles idées à ajouter au site, avec d'autant plus de travail derrière. Pour réaliser et administrer tout le système, nous avons une fourchette de plus ou moins 2 mois et demi, ce qui à première vu paraît aussi bref que le premier lot l'est.

\subsection{Contrainte monétaire}
\label{Contrainte monétaire}

En ce qui concerne le budget des dépenses en vue de la réalisation de ce projet, nous nous sommes fixés comme objectif de ne rien dépenser. Nous avons donc dû trouver des alternatives à des services de stockage de données comme DynamoDB, qui est payant.

\subsection{Contrainte législative}

Nous avons ajouté des pages de termes et de conditions. Cette page est née d'une certaine contrainte législative liée à la GDPR\parencite{Ref3} (RGPD), et sera explicitée plus en détails dans le second lot.

\section{Outils de gestion de projet et d'équipe}
\label{Outils de gestion de projet et d’équipe}

\subsubsection*{GanttProject}
GanttProject est un logiciel sous GNU GPL (licence publique de distribution), existant depuis 2003. Il fait parti des pionniers dans ce type d’application.

Comme son nom l'indique, il va nous permettre de réaliser un diagramme de GANTT afin de suivre notre avancée au fil du temps.

\subsubsection*{Visual Paradigm}
Visual Paradigm est un logiciel gratuit mis au point en Java le 20 juin 2002 par Visual Paradigm International Ltd.

Ce logiciel va nous permettre de facilement créer tous les diagrammes UML/SYSML qui vont nous être utiles.

\subsubsection*{Git}
Git est un logiciel connu de tout développeur informatique, lancé en avril 2005. Utilisant les protocoles HTTP(S) / SSH ainsi que celui spécifique à Git, sa fonction première est la gestion de versions.

Ce dernier, couplé à GitHub, va nous servir d'outil de versionning de code.

\subsubsection*{Trello}
Trello est un logiciel de gestion de projet en collaboration en ligne lancé en septembre 2011.

Le kanban de Trello va nous servir à suivre notre avancée en termes de jalons et de tâches.

\section{Descriptif du projet}

VroomMates, notre plateforme de covoiturage, est destinée à toute personne recherchant un trajet à partager, d'un point A à un point B, à une date donnée. Comme mentionné dans le rapport du premier lot, notre application a un but d'internationalisation : elle a donc été traduite intégralement en anglais, la langue universelle, afin d'être compréhensible par la majorité des personnes dans le monde. De plus, nous avons décidé de nous concentrer sur la partie mobile, car c'est pour nous le cas d'utilisation premier.

Sur la plateforme, il existe quatre types d'utilisateurs, chacun avec son propre accès aux différentes fonctionnalités disponibles:

\subparagraph{Visiteur}

On définit un visiteur comme tout personne non-connectée sur la plateforme. Il peut concrètement visiter les pages où il n'est pas demandé d'authentification, comme la recherche et les détails d'un voyage. Cependant, il lui sera impossible d'interagir d'avantage avec la commande, le voyage ou le conducteur.
Bien sûr, ce dernier aura accès aux termes et conditions, à l'accueil ainsi qu'aux pages d'inscription et de connexion.

\subparagraph{Passager}

On considère un passager comme tel à partir du moment où le visiteur a créé son compte et s'est connecté à celui-ci. Dès lors, il lui est possible tout un tas de nouvelles fonctionnalités, comme pouvoir réserver un voyage ou encore noter le conducteur avec qui il a voyagé. De plus, il pourra renseigner certains champs de son profil, tout en changeant par exemple sa photo de profil, ses préférences de conduite, etc. Le passager aura un accès aux notes précédemment données et celles reçues, ainsi qu'à son historique de trajets. Il pourra aussi, après une confirmation manuelle des administrateurs du site, pouvoir devenir conducteur. 

\subparagraph{Conducteur}

En plus de toutes les fonctionnalités précédemment énumérées, un conducteur peut créer ou encore gérer ses voyages. Ces utilisateurs seront indiqués d'un petit volant de voiture à côté du prénom.

\subparagraph{Administrateur}

L'administrateur a la possibilité en plus d'afficher une liste d'utilisateurs et de les rechercher en fonction de leur nom et / ou prénom. En tant que "entité de modération", il leur est possible de bannir/débannir un membre. Une fois un compte banni, il ne sera plus possible d'interagir avec des trajets ou des utilisateurs.
